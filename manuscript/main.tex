\documentclass[10pt]{article}


\usepackage{geometry}
\geometry{left=1.7cm,right=1.7cm, top=1.7cm, bottom=1.7cm}

\usepackage{multicol}
\setlength{\columnsep}{1cm}

\usepackage{nopageno}


\usepackage{newtxtext}
\usepackage{newtxmath}
\usepackage{courier}

\usepackage[brazil]{babel}

\usepackage{hyperref}

\usepackage{booktabs}
\usepackage{adjustbox}

\usepackage[ruled]{algorithm2e}

% ----------------------------
% Updating section headers
\usepackage{titlesec}

\renewcommand{\thesection}{\Roman{section}.}
\titleformat{\section}
{\scshape\centering}{\thesection}{1em}{}

\renewcommand{\thesubsection}{\Alph{subsection}.}
\titleformat{\subsection}
{\itshape}{\thesubsection}{1em}{}

% ----------------------------
% Updating title and authors
\usepackage{titling}
\usepackage{authblk}
\renewcommand\Authand{, }

\pretitle{
    \begin{center}
        \LARGE
}
\posttitle{
    \end{center}
}

\preauthor{
    \fontsize{11}{11}
    \begin{center}
}
\postauthor{
    \end{center}
    \fontsize{10}{10}
}

\title{Algoritmos de Coloração\vspace{-1ex}}

\author[1]{Lucas Guido}
\author[2]{Caio Stoduto\fontsize{10}{10}\vspace{-1ex}}
\affil[1]{\fontsize{10}{10}Universidade Federal do ABC, Santo André -- SP, Brasil  \authorcr
\fontsize{9}{9}\texttt{lucas.guido@aluno.ufabc.edu.br}\fontsize{10}{10}}
\affil[2]{\fontsize{10}{10}Universidade Federal do ABC, Santo André -- SP, Brasil \authorcr
\fontsize{9}{9}\texttt{caio.stoduto@aluno.ufabc.edu.br}\fontsize{10}{10}}

\date{}

\begin{document}

\maketitle

\noindent
\small
\textit{\textbf{Resumo}}

\noindent
Bla bla

\mdseries
\normalsize
\indent

\begin{multicols*}{2}
\section{Introdução}

\section{Algoritmos}


% ------------------------------------------
% First Fit Algorithm

\begin{algorithm}[H]
\caption{First Fit (FF)}
\label{alg:ff}

\KwData{Grafo $G(V,E)$}
\KwResult{Coloração de $G$}

$colors \gets \emptyset$\;
$v \gets V[1]$;

\end{algorithm}


% ------------------------------------------
% Welsh Powell Algorithm

\begin{algorithm}[H]
\caption{Welsh Powell (WP)}
\label{alg:wp}

\KwData{Grafo $G(V,E)$}
\KwResult{Coloração de $G$}

$colors \gets \emptyset$\;
$v \gets V[1]$;

\end{algorithm}


% ------------------------------------------
% Largest Degree Ordering Algorithm

\begin{algorithm}[H]
\caption{Largest Degree Ordering (LDO)}
\label{alg:ldo}

\KwData{Grafo $G(V,E)$}
\KwResult{Coloração de $G$}

$colors \gets \emptyset$\;
$v \gets V[1]$;

\end{algorithm}


% ------------------------------------------
% Incidence Degree Ordering Algorithm

\begin{algorithm}[H]
\caption{Incidence Degree Ordering (IDO)}
\label{alg:ido}

\KwData{Grafo $G(V,E)$}
\KwResult{Coloração de $G$}

$colors \gets \emptyset$\;
$v \gets V[1]$;

\end{algorithm}


% ------------------------------------------
% Degree of Saturation Algorithm

\begin{algorithm}[H]
\caption{Degree of Saturation (DSATUR)}
\label{alg:dsatur}

\KwData{Grafo $G(V,E)$}
\KwResult{Coloração de $G$}

$colors \gets \emptyset$\;
$v \gets V[1]$;

\end{algorithm}


% ------------------------------------------
% Recursive Largest First Algorithm

\begin{algorithm}[H]
\caption{Recursive Largest First (RLF)}
\label{alg:rlf}

\KwData{Grafo $G(V,E)$}
\KwResult{Coloração de $G$}

$colors \gets \emptyset$\;
$v \gets V[1]$;

\end{algorithm}


\section{Experimentos computacionais}
\begin{table*}
    \caption{Experimentos computacionais}
    \begin{adjustbox}{width=\textwidth}%
    \small
    \centering
    \begin{minipage}[b]{\textwidth}
        \centering
\begin{tabular}{@{} lrrrrrrrrrrrrrr @{}}
\toprule
& & & \multicolumn{2}{c}{FF} & \multicolumn{2}{c}{WP} & \multicolumn{2}{c}{LDO} & \multicolumn{2}{c}{IDO} & \multicolumn{2}{c}{DSATUR} & \multicolumn{2}{c}{RLF}\\
\cmidrule(lr){4-5}\cmidrule(lr){6-7}\cmidrule(lr){8-9}\cmidrule(lr){10-11}\cmidrule(lr){12-13}\cmidrule(lr){14-15}
Grafo & Vértices & Arestas & K & T (ms) & K & T (ms) & K & T (ms) & K & T (ms) & K & T (ms) & K & T (ms) \\
\midrule
C2000.5 & 2000 & 999836 & 226 & 19.98 & 222 & 7.61 & 222 & 17.55 & 222 & 22.69 & 221 & 20.27 & NaN & NaN \\
C4000.5 & 4000 & 4000268 & 402 & 136.68 & 396 & 28.22 & 396 & 104.60 & 403 & 131.21 & 402 & 119.22 & NaN & NaN \\
dsjc1000.1 & 1000 & 49629 & 31 & 0.24 & 30 & 0.76 & 30 & 0.29 & 30 & 1.29 & 30 & 1.33 & NaN & NaN \\
dsjc1000.5 & 1000 & 249826 & 127 & 3.02 & 124 & 2.00 & 124 & 3.12 & 127 & 4.49 & 127 & 3.85 & NaN & NaN \\
dsjc1000.9 & 1000 & 449449 & 321 & 13.28 & 310 & 3.20 & 310 & 13.72 & 318 & 15.34 & 309 & 14.54 & NaN & NaN \\
dsjc250.5 & 250 & 31336 & 43 & 0.12 & 40 & 0.13 & 40 & 0.12 & 40 & 0.23 & 41 & 0.22 & NaN & NaN \\
dsjc500.1 & 500 & 12458 & 20 & 0.06 & 18 & 0.20 & 18 & 0.07 & 19 & 0.29 & 19 & 0.27 & NaN & NaN \\
dsjc500.5 & 500 & 62624 & 72 & 0.48 & 69 & 0.48 & 69 & 0.55 & 70 & 0.76 & 71 & 0.75 & NaN & NaN \\
dsjc500.9 & 500 & 224874 & 175 & 1.94 & 168 & 0.83 & 168 & 2.10 & 174 & 2.85 & 176 & 2.91 & NaN & NaN \\
dsjr500.1c & 500 & 121275 & 109 & 0.58 & 99 & 0.51 & 99 & 1.13 & 101 & 1.46 & 97 & 1.45 & NaN & NaN \\
dsjr500.5 & 500 & 58862 & 143 & 0.99 & 133 & 0.72 & 133 & 1.33 & 129 & 1.53 & 133 & 1.64 & NaN & NaN \\
flat1000-50-0 & 1000 & 245000 & 126 & 2.85 & 121 & 2.05 & 121 & 2.85 & 122 & 3.97 & 122 & 3.83 & NaN & NaN \\
flat1000-60-0 & 1000 & 245830 & 125 & 2.99 & 124 & 2.01 & 124 & 2.99 & 123 & 4.03 & 124 & 3.90 & NaN & NaN \\
flat1000-76-0 & 1000 & 246708 & 122 & 2.80 & 123 & 1.99 & 123 & 2.87 & 125 & 4.31 & 123 & 3.72 & NaN & NaN \\
flat300-28-0 & 300 & 21695 & 46 & 0.18 & 46 & 0.19 & 46 & 0.17 & 49 & 0.26 & 46 & 0.24 & NaN & NaN \\
latin-square & 900 & 307350 & 213 & 3.44 & 155 & 1.46 & 155 & 5.02 & 147 & 6.21 & 158 & 5.68 & NaN & NaN \\
le450-25c & 450 & 17343 & 37 & 0.11 & 31 & 0.24 & 31 & 0.15 & 30 & 0.32 & 32 & 0.31 & NaN & NaN \\
le450-25d & 450 & 17425 & 35 & 0.11 & 30 & 0.26 & 30 & 0.14 & 31 & 0.32 & 33 & 0.33 & NaN & NaN \\
r1000.1c & 1000 & 485090 & 138 & 2.14 & 117 & 1.31 & 117 & 3.70 & 124 & 5.25 & 116 & 4.64 & NaN & NaN \\
r1000.5 & 1000 & 238267 & 275 & 6.80 & 259 & 3.04 & 259 & 9.68 & 250 & 10.03 & 253 & 10.03 & NaN & NaN \\
r250.5 & 250 & 14849 & 79 & 0.20 & 71 & 0.17 & 71 & 0.25 & 69 & 0.28 & 71 & 0.28 & NaN & NaN \\
\bottomrule
\end{tabular}
\end{minipage}
\end{adjustbox}
\end{table*}


\section{Conclusão}
\end{multicols*}

\end{document}