\documentclass[10pt]{article}


\usepackage{geometry}
\geometry{left=1.7cm,right=1.7cm, top=1.7cm, bottom=1.7cm}

\usepackage{multicol}
\setlength{\columnsep}{1cm}

\usepackage{nopageno}


\usepackage{newtxtext}
\usepackage{newtxmath}
\usepackage{courier}

\usepackage[brazil]{babel}

\usepackage{hyperref}

\usepackage{lipsum}

\usepackage[ruled]{algorithm2e}

% ----------------------------
% Updating section headers
\usepackage{titlesec}

\renewcommand{\thesection}{\Roman{section}.}
\titleformat{\section}
{\scshape\centering}{\thesection}{1em}{}

\renewcommand{\thesubsection}{\Alph{subsection}.}
\titleformat{\subsection}
{\itshape}{\thesubsection}{1em}{}

% ----------------------------
% Updating title and authors
\usepackage{titling}
\usepackage{authblk}
\renewcommand\Authand{, }

\pretitle{
    \begin{center}
        \LARGE
}
\posttitle{
    \end{center}
}

\preauthor{
    \fontsize{11}{11}
    \begin{center}
}
\postauthor{
    \end{center}
    \fontsize{10}{10}
}

\title{Algoritmos de Coloração\vspace{-1ex}}

\author[1]{Lucas Guido}
\author[2]{Caio Stoduto\fontsize{10}{10}\vspace{-1ex}}
\affil[1]{\fontsize{10}{10}Universidade Federal do ABC, Santo André -- SP, Brasil  \authorcr
\fontsize{9}{9}\texttt{lucas.guido@aluno.ufabc.edu.br}\fontsize{10}{10}}
\affil[2]{\fontsize{10}{10}Universidade Federal do ABC, Santo André -- SP, Brasil \authorcr
\fontsize{9}{9}\texttt{caio.stoduto@aluno.ufabc.edu.br}\fontsize{10}{10}}

\date{}

\begin{document}

\maketitle

\noindent
\small
\textit{\textbf{Resumo}}

\noindent
Bla bla

\mdseries
\normalsize
\indent

\begin{multicols*}{2}
\section{Introdução}
\lipsum


\section{Algoritmos}


% ------------------------------------------
% First Fit Algorithm

\begin{algorithm}[H]
\caption{First Fit (FF)}
\label{alg:ff}

\KwData{Grafo $G(V,E)$}
\KwResult{Coloração de $G$}

$colors \gets \emptyset$\;
$v \gets V[1]$;

\end{algorithm}


% ------------------------------------------
% Welsh Powell Algorithm

\begin{algorithm}[H]
\caption{Welsh Powell (WP)}
\label{alg:wp}

\KwData{Grafo $G(V,E)$}
\KwResult{Coloração de $G$}

$colors \gets \emptyset$\;
$v \gets V[1]$;

\end{algorithm}


% ------------------------------------------
% Largest Degree Ordering Algorithm

\begin{algorithm}[H]
\caption{Largest Degree Ordering (LDO)}
\label{alg:ldo}

\KwData{Grafo $G(V,E)$}
\KwResult{Coloração de $G$}

$colors \gets \emptyset$\;
$v \gets V[1]$;

\end{algorithm}


% ------------------------------------------
% Incidence Degree Ordering Algorithm

\begin{algorithm}[H]
\caption{Incidence Degree Ordering (IDO)}
\label{alg:ido}

\KwData{Grafo $G(V,E)$}
\KwResult{Coloração de $G$}

$colors \gets \emptyset$\;
$v \gets V[1]$;

\end{algorithm}


% ------------------------------------------
% Degree of Saturation Algorithm

\begin{algorithm}[H]
\caption{Degree of Saturation (DSATUR)}
\label{alg:dsatur}

\KwData{Grafo $G(V,E)$}
\KwResult{Coloração de $G$}

$colors \gets \emptyset$\;
$v \gets V[1]$;

\end{algorithm}


% ------------------------------------------
% Recursive Largest First Algorithm

\begin{algorithm}[H]
\caption{Recursive Largest First (RLF)}
\label{alg:rlf}

\KwData{Grafo $G(V,E)$}
\KwResult{Coloração de $G$}

$colors \gets \emptyset$\;
$v \gets V[1]$;

\end{algorithm}


\section{Experimentos computacionais}
    \lipsum

\section{Conclusão}

\end{multicols*}
\end{document}